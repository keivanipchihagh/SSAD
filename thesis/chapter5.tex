\chapter{نتیجه گیری، جمع ‌بندی و پیشنهادات}

\section{جمع‌بندي و نتيجه‌گيري}

در این پژوهش به بررسی معماری‌های مختلف برای حل مسئله تقسیم‌بندی معنایی تصاویر در حوزه خودرو‌های خودران با استفاده از روش های یادگیری عمیق پرداختیم. همانطور که در مقدمه به طور مفصل‌تر به آن پرداخته شد، مدل‌های مورد استفاده برای این حوزه بخصوص، علاوه بر دقت نیاز به سرعت عمل بالا نیز دارند که معیار مهمی در ارزیابی نهایی آنها است.

در فصل دوم، به مطالعه مفاهیم پرتکرار این حوزه پرداخته و معماری‌های مورد استفاده، نظیر معماری رمزگذار-رمزگشا، را معرفی کردیم که کاربرد گسترده ای در مدل‌های مطرح برای این حوزه دارد و به جزئیات آن پرداختیم. سپس چندین معماری مطرح در حوزه تقسیم‌بندی معنایی را معرفی کردیم که از دقت بالایی برخوردار بوده، اما عملکرد خوبی در پردازش آنی ندارند که مشخصه مهمی در ارزیابی نهایی است.

در فصل سوم، به طور عمیق وارد معماری منحصر به فرد مدل‌های پیشنهادی و اجزای کلیدی آنها شدیم و نقاط ضعف و قوت هر یک را بررسی و مقایسه کردیم. با وجود مدل‌های متعدد در حوزه تقسیم‌بندی معنایی، می‌توان گفت اکثر مدل‌ها از معماری رمزگذار-رمزگشا و یا مشابه آن استفاده می‌کنند تا بتوانند عملکرد بهینه‌تری در سرعت پردازش بدست آورند. متوجه شدیم با تغییر بر روی اجزای این معماری، مانند تعداد لایه ها، نوع توابع‌فعال‌ساز، حذف نرمال‌سازی، و موارد مشابه می‌توان بر تعداد وزن‌های مورد نیاز و میزان محاسبات لازم را کاهش داد و در نهایت بر روی سرعت پردازش تاثیر مثبت گذاشت. همچنین می‌توان با تغییر در معماری مانند افزودن پرش، ترکیب داده‌های لایه‌ها و استفاده از معماری دو-شاخه بر، دقت و کیفیت تصویر بازسازی شده را بهبود داد.

در فصل چهارم، آزمایش‌های و نتایج، به آموزش و ارزیابی معماری‌های مطرح شده پرداختیم. ارزیابی‌های صورت گرفته نه تنها بر روی دقت و شاخص بازسازی تصاویر بود، بلکه بر روی سرعت پردازش و مصرف منابع مدل‌ها نیز تمرکز داشتیم. طی مقایسه عملکرد مدل‌ها متوجه شدیم استفاده از معماری دو-شاخه در معماری اولیه رمزگذار-رمزگشا تاثیر مثبتی بر روی سرعت پردازش می‌گذارد و در عین حال دقت مدل دچار نوسان چندانی نمی‌شود که مطلوب ما است.

\section{پیشنهادات و کار‌های آتی}

روش‌های مورد بحث و بررسی قرار گرفته داخل این پروژه همگی بر روی تصاویر تمرکز داشته‌اند؛ به گونه‌ای که برای پردازش ویدیو، هر فریم به تنهایی و مجزا از فریم‌های پیشین پردازش می‌شود. چالش روش فعلی آن است که امکان تغییر دسته‌بندی ها بین دو تصویر متوالی در یک ویدیو وجود دارد و سازوکاری برای کاهش و یا جلو‌گیری از آن نداریم. مدل‌های نوین‌تر تقسیم‌بندی معنایی ویدیویی
\LTRfootnote{Video semantic segmentation}
نظیر 
\verb*|TMANET|
\cite{wang2021temporal}
\LTRfootnote{Temporal Memory Attention Network}
به حل این مشکل می‌پردازند. در این گونه مدل های، یک یا چند تصویر گذشته بر روی تقسیم‌بندی معنایی تصویر بعدی، به صورت وزن‌دار، تاثیرگذار هستند و بنابراین امکان تغییر ناگهانی یک دسته به دلیل خطای مدل و یا شرایط جدید محیطی کاهش میابد. هرچند استفاده از این گونه مدل‌های ویدیووی در حوزه خودرو‌های خودران مانند مدل‌های تقسیم‌بندی تصویر مرسوم نیست، تمرکز بیشتر بر روی این مدل ها و بهینه‌سازی آنها برای پردازش سریع‌تر پیشنهاد می‌شود.
