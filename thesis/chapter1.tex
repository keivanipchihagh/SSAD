\chapter{مقدمه}

\section{شرح مسأله}
در دهه اخیر، پیشرفت‌های چشمگیری در زمینه هوش مصنوعی و یادگیری عمیق، به ویژه در حوزه پردازش تصویر و استفاده از آنها برای بهبود عملکرد تصمیم‌گیری در خودروهای خودران، انقلابی در روند توسعه و بهینه‌سازی فناوری در این زمینه ایجاد کرده است. شبکه‌های عصبی عمیق
\LTRfootnote{Deep neural networks}
به دلیل قابلیت‌هایی که از طریق شبکه‌های عصبی کانولوشنی
\LTRfootnote{convolutional neural networks}
\cite{o2015introduction}
فراهم می‌آید، امکاناتی را برای خودروها فراهم می‌کند که پیش از این غیرقابل تصور بوده است.

با این وجود، یکی از چالش‌های بزرگ در مسیر توسعه خودروهای خودران، توانایی فهم و تفسیر دقیق محیط اطراف و اشیاء موجود در تصویر است. برای حل این چالش معمولاً از روش‌های متنوعی استفاده می‌شود که یکی از روش‌های مهم در این زمینه، تقسیم بندی معنایی نامیده می‌شود. در این روش، تمامی سلول‌‌های
\LTRfootnote{Pixel}
تصویری موجود به دسته‌هایی از پیش تعیین شده تخصیص داده می‌شوند. خودرو باید توانایی آن را داشته باشد تا اطلاعات دریافتی از محیط را با سرعت در لحظه
\LTRfootnote{Real-time}
به دسته‌های مختلف مانند خیابان، پیاده‌رو، خودروها، چراغ راهنما و غیره تقسیم بندی کرده و به هر دسته یک رنگ مخصوص که اسطلاحا به آن رنگبندی تقسیم‌بندی
\LTRfootnote{Segmentation color}
گفته می‌شود اختصاص دهد.

بدون شک، تقسیم‌بندی معنایی محیط برای خودروهای خودران امری بسیار اساسی و حیاتی است. اطلاعات دقیق و صحیح در مورد محیط اطراف، به سیستم‌های خودران امکان می‌دهد تا تصمیمات صحیح و ایمن را در مسیر حرکت خود اتخاذ کنند. این اطلاعات، پایه‌ای برای عملکرد امن و کارآمد این خودروهای خودران است. در عین اهمیت داشتن دقت بالا در انجام این امر، پردازش در لحظه نیز حائز اهمیت است. زیرا تنها داشتن دقت بالا بدون توانایی پردازش سریع و به موقع، در مسائلی که نیاز به پردازش آنی دارند ناکارآمد خواهد بود. به عبارتی دیگر، دقت بالا و سرعت پردازش به‌طور همزمان، می‌توانند به عنوان دو عامل اساسی و مکمل، عملکرد بهینه سیستم خودران را فراهم کنند.

\section{اهداف پروژه}
ادر بخش ابتدایی از پروژه، به توضیح مقدمه‌ای بر چگونگی انجام تقسیم‌بندی معنایی تصاویر پرداخته و سپس به بررسی روش‌های تقسیم‌بندی معنایی بااستفاده از یادگیری عمیق که برای استفاده در حوزه تصویربرداری پزشکی طراحی شده‌اند
\LTRfootnote{Medical Imaging}
می‌پردازیم. تمرکز ما در این بخش بررسی مدل‌هایی است که از دقت بالایی برخوردار هستند و سرعت عمل به عنوان یک مشخصه ثانویه مطرح نمی‌شود، زیرا هدف این مدل‌ها در صنعت پزشکی، تشخیص درست و دقیق اجزای موجود در تصویر است که سرعت حائز اهمیت چندانی نیست.
در ادامه، به بررسی روش‌های یادگیری عمیق برای تقسیم‌بندی معنایی در حوزه خودروهای خودران با تمرکز بر پردازش در لحظه پرداخته و هدف ما دقت بالا و سرعت عمل بهینه متناسب با این مسئله است. با توجه به نیازهای خاص این حوزه، ما به دنبال راهکارها و معماری‌هایی هستیم که بهبود دقت و سرعت عمل سیستم‌های خودران را به هدف داشته و در نتیجه، ایمنی و کارایی این سیستم‌ها را بهبود بخشند.
در پایان، به جمع‌بندی و نتیجه‌گیری مطالب بدست آمده در این پروژه پرداخته و مروری بر سیر انجام پروژه و معماری‌های مطرح شده خواهیم داشت و سپس پیشنهاداتی برای کار‌های آینده که می‌تواند به بهبود وضعیت فعلی کمک کند، مطرح می‌کنیم.

\section{ساختار گزارش}
در فصل ابتدایی این گزارش، مقدمه‌ای بر روی مسأله مطرح شده در این پروژه و شرح کلی از اهداف و محتوای گزارش ارائه شد. در فصل دوم، به طور مفصل به مفاهیم مرتبط و چگونگی پیاده‌سازی این مسأله پرداخته و سپس به بررسی معماری رمزگذار-رمزگشا و مدل‌هایی که از این معماری استفاده می‌کنند پرداخته خواهد شد. در فصل سوم، به مدل‌های طراحی شده برای مساله خاص تقسیم‌بندی معنایی در خودروهای خودران اشاره شده و سپس مدل‌های پیشنهادی این پژوهش انتخاب و به طور مفصل توضیح داده خواهد شد. در فصل چهارم، آزمایش‌ها، نتایج و ارزیابی‌های انجام شده بر روی کیفیت و کارایی مدل‌ها مورد بحث و بررسی قرار خواهند گرفت. در فصل پنجم، به عنوان فصل پایانی، جمع‌بندی نکات گزارش و پیشنهاد‌هایی برای کارهای آینده به منظور بهبود عملکرد و کارایی در این حوزه خواهد شد.