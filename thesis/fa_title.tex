%% -!TEX root = AUTthesis.tex
% در این فایل، عنوان پایان‌نامه، مشخصات خود، متن تقدیمی‌، ستایش، سپاس‌گزاری و چکیده پایان‌نامه را به فارسی، وارد کنید.
% توجه داشته باشید که جدول حاوی مشخصات پروژه/پایان‌نامه/رساله و همچنین، مشخصات داخل آن، به طور خودکار، درج می‌شود.
%%%%%%%%%%%%%%%%%%%%%%%%%%%%%%%%%%%%
% دانشکده، آموزشکده و یا پژوهشکده  خود را وارد کنید
\faculty{دانشکده مهندسی کامپیوتر}
% گرایش و گروه آموزشی خود را وارد کنید
\department{}
% عنوان پایان‌نامه را وارد کنید
\fatitle{تقسیم‌بندی معنایی برای ماشین‌های خودران با استفاده از یادگیری عمیق
\\[.75 cm]
}
% نام استاد(ان) راهنما را وارد کنید
\firstsupervisor{دکتر احسان ناظرفرد}
%\secondsupervisor{استاد راهنمای دوم}
% نام استاد(دان) مشاور را وارد کنید. چنانچه استاد مشاور ندارید، دستور پایین را غیرفعال کنید.
%\firstadvisor{استاد مشاور اول}
%\secondadvisor{استاد مشاور دوم}
% نام نویسنده و نام خانوادگیرا وارد کنید
\name{کیوان}

\surname{ ایپچی حق }
%%%%%%%%%%%%%%%%%%%%%%%%%%%%%%%%%%
\thesisdate{فروردین ۱۴۰۳}

% چکیده پایان‌نامه را وارد کنید
\fa-abstract{
خودروهای خودران
\LTRfootnote{Self-driving cars}
به منظور اتخاذ تصمیمات آگاهانه و مسیریابی ایمن در محیط‌های مختلف، نیازمند درک دقیقی از اشیاء اطراف خود هستند. تقسیم‌بندی معنایی
\LTRfootnote{Semantic segmentation}
از ابتدایی‌ترین مراحل در فرایند تجزیه و تحلیل تصاویر و استخراج اطلاعات مفید آن به منظور تصمیم‌گیری در اینگونه سیستم‌ها است که نقش حیاتی در تشخیص اشیاء محیط دارد و این امکان را بوجود می‌آورد تا به طور دقیق اشیاء مختلف از جمله جاده‌ها، عابران پیاده، خودروهای دیگر و موانع شناسایی شوند. روش‌های یادگیری عمیق
\LTRfootnote{Deep learning}
بهبود قابل توجهی در تقسیم‌بندی معنایی تصاویر به وجود آورده‌اند، به گونه‌ای که از عملکرد برتری نسبت به روش‌های سنتی برخوردار هستند. این پروژه به بررسی پیشرفت‌های اخیر در زمینه تقسیم‌بندی معنایی تصاویر برای خودروهای خودران با استفاده از روش‌های یادگیری عمیق می‌پردازد. ما معماری‌های مختلف یادگیری عمیق را در مسئله تقسیم‌بندی معنایی سریع
\LTRfootnote{Fast semantic segmentation}
مورد مطالعه قرار داده، معماری‌های مختلف را مقایسه نموده و نقاط قوت و ضعف آنها را برای مسئله خاص خودروهای خودران بررسی می‌کنیم. علاوه بر این، مجموعه‌داده‌های مورد استفاده برای آموزش و ارزیابی مدل‌های تقسیم‌بندی معنایی در این حوزه مورد بررسی قرار گرفته و از آنها برای ارزیابی مدل‌های یادگیری عمیق مختلف استفاده می‌شود. در پایان، جمع‌بندی بر روی مدل‌های مورد بررسی قرار گرفته خواهیم داشت و پیشنهاداتی برای پژوهش‌های آینده در جهت بهبود پایداری، کارایی و قابلیت عمومی سیستم‌های تقسیم‌بندی معنایی در لحظه مبتنی بر یادگیری عمیق برای خودروهای خودران ارائه می‌شود.
}

% کلمات کلیدی پایان‌نامه را وارد کنید
\keywords{هوش مصنوعی، خودرو‌های خودران، یادگیری عمیق، تقسیم بندی معنایی، تقسیم‌بندی معنایی سریع تصاویر}



\AUTtitle
%%%%%%%%%%%%%%%%%%%%%%%%%%%%%%%%%%
\vspace*{7cm}
\thispagestyle{empty}
\begin{center}
\includegraphics[height=5cm,width=12cm]{Images/besm.jpg}
\end{center}